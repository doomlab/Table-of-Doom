\documentclass[english,man]{apa6}

\usepackage{amssymb,amsmath}
\usepackage{ifxetex,ifluatex}
\usepackage{fixltx2e} % provides \textsubscript
\ifnum 0\ifxetex 1\fi\ifluatex 1\fi=0 % if pdftex
  \usepackage[T1]{fontenc}
  \usepackage[utf8]{inputenc}
\else % if luatex or xelatex
  \ifxetex
    \usepackage{mathspec}
    \usepackage{xltxtra,xunicode}
  \else
    \usepackage{fontspec}
  \fi
  \defaultfontfeatures{Mapping=tex-text,Scale=MatchLowercase}
  \newcommand{\euro}{€}
\fi
% use upquote if available, for straight quotes in verbatim environments
\IfFileExists{upquote.sty}{\usepackage{upquote}}{}
% use microtype if available
\IfFileExists{microtype.sty}{\usepackage{microtype}}{}

% Table formatting
\usepackage{longtable, booktabs}
\usepackage{lscape}
% \usepackage[counterclockwise]{rotating}   % Landscape page setup for large tables
\usepackage{multirow}		% Table styling
\usepackage{tabularx}		% Control Column width
\usepackage[flushleft]{threeparttable}	% Allows for three part tables with a specified notes section
\usepackage{threeparttablex}            % Lets threeparttable work with longtable

% Create new environments so endfloat can handle them
% \newenvironment{ltable}
%   {\begin{landscape}\begin{center}\begin{threeparttable}}
%   {\end{threeparttable}\end{center}\end{landscape}}

\newenvironment{lltable}
  {\begin{landscape}\begin{center}\begin{ThreePartTable}}
  {\end{ThreePartTable}\end{center}\end{landscape}}

  \usepackage{ifthen} % Only add declarations when endfloat package is loaded
  \ifthenelse{\equal{\string man}{\string man}}{%
   \DeclareDelayedFloatFlavor{ThreePartTable}{table} % Make endfloat play with longtable
   % \DeclareDelayedFloatFlavor{ltable}{table} % Make endfloat play with lscape
   \DeclareDelayedFloatFlavor{lltable}{table} % Make endfloat play with lscape & longtable
  }{}%



% The following enables adjusting longtable caption width to table width
% Solution found at http://golatex.de/longtable-mit-caption-so-breit-wie-die-tabelle-t15767.html
\makeatletter
\newcommand\LastLTentrywidth{1em}
\newlength\longtablewidth
\setlength{\longtablewidth}{1in}
\newcommand\getlongtablewidth{%
 \begingroup
  \ifcsname LT@\roman{LT@tables}\endcsname
  \global\longtablewidth=0pt
  \renewcommand\LT@entry[2]{\global\advance\longtablewidth by ##2\relax\gdef\LastLTentrywidth{##2}}%
  \@nameuse{LT@\roman{LT@tables}}%
  \fi
\endgroup}


\ifxetex
  \usepackage[setpagesize=false, % page size defined by xetex
              unicode=false, % unicode breaks when used with xetex
              xetex]{hyperref}
\else
  \usepackage[unicode=true]{hyperref}
\fi
\hypersetup{breaklinks=true,
            pdfauthor={},
            pdftitle={LAB: Linguistic Annotated Bibliography -- A searchable portal for normed database information},
            colorlinks=true,
            citecolor=blue,
            urlcolor=blue,
            linkcolor=black,
            pdfborder={0 0 0}}
\urlstyle{same}  % don't use monospace font for urls

\setlength{\parindent}{0pt}
%\setlength{\parskip}{0pt plus 0pt minus 0pt}

\setlength{\emergencystretch}{3em}  % prevent overfull lines

\ifxetex
  \usepackage{polyglossia}
  \setmainlanguage{}
\else
  \usepackage[english]{babel}
\fi

% Manuscript styling
\captionsetup{font=singlespacing,justification=justified}
\usepackage{csquotes}
\usepackage{upgreek}

 % Line numbering
  \usepackage{lineno}
  \linenumbers


\usepackage{tikz} % Variable definition to generate author note

% fix for \tightlist problem in pandoc 1.14
\providecommand{\tightlist}{%
  \setlength{\itemsep}{0pt}\setlength{\parskip}{0pt}}

% Essential manuscript parts
  \title{LAB: Linguistic Annotated Bibliography -- A searchable portal for normed
database information}

  \shorttitle{Linguistic Bibliography}


  \author{Erin M. Buchanan\textsuperscript{1}~\& Kathrene D. Valentine\textsuperscript{2}}

  % \def\affdep{{"", ""}}%
  % \def\affcity{{"", ""}}%

  \affiliation{
    \vspace{0.5cm}
          \textsuperscript{1} Missouri State University\\
          \textsuperscript{2} University of Missouri  }

  \authornote{
    Erin M. Buchanan is an Associate Professor of Quantitative Psychology at
    Missouri State University. K. D. Valentine is a Ph.D.~candidate at the
    University of Missouri.
    
    Correspondence concerning this article should be addressed to Erin M.
    Buchanan, 901 S. National Ave, Springfield, MO 65897. E-mail:
    \href{mailto:erinbuchanan@missouristate.edu}{\nolinkurl{erinbuchanan@missouristate.edu}}
  }


  \abstract{In the era of big data, psycholinguistic research is flourishing with
numerous publications advancing our knowledge of word characteristics
and ways to study them. This article presents the Linguistic Annotated
Bibliography (LAB) as a searchable web portal to quickly and easily
access reliable database norms, related programs, and variable
calculations. These publications (N = 561) were coded by language,
number of stimuli, stimuli type (i.e.~words, pictures, symbols),
keywords (i.e.~frequency, semantics, valence), and other useful
information. This tool not only allows researchers to search for the
specific type of stimuli needed for experiments, but also permits the
exploration of publication trends across 100 years of research. Details
about the portal creation and use are outlined, as well as various
analyses of change in publication rates and keywords. In general,
advances in computation power have allowed for the increase in dataset
size in the recent decades, in addition to an increase in the number of
linguistic variables provided in each publication.}
  \keywords{database, stimuli, online portal, megastudy, trends \\

    \indent Word count: X
  }





\usepackage{amsthm}
\newtheorem{theorem}{Theorem}
\newtheorem{lemma}{Lemma}
\theoremstyle{definition}
\newtheorem{definition}{Definition}
\newtheorem{corollary}{Corollary}
\newtheorem{proposition}{Proposition}
\theoremstyle{definition}
\newtheorem{example}{Example}
\theoremstyle{definition}
\newtheorem{exercise}{Exercise}
\theoremstyle{remark}
\newtheorem*{remark}{Remark}
\newtheorem*{solution}{Solution}
\begin{document}

\maketitle

\setcounter{secnumdepth}{0}



The advance of computational ability and the Internet have propelled
research into an era of so-called big data that have interesting
implications for the field of psycholinguistics, as well as other
experimental areas that use normed stimuli for their research.
Traditionally, stimuli used for experimental research in
psycholinguistics were first examined through small pilot studies that
were then used in many subsequent projects. While economic, that
selection procedure's results could be potentially misleading as a
factor of the stimuli, rather than experimental manipulation. This
limitation can potentially be tied to a lack of funding, time, computing
power, or even interest in studying phenomena at the stimuli level. Now,
we have the capability to collect, analyze, and publish large datasets
for research into memory models (Cree, McRae, \& McNorgan, 1999; Moss,
Tyler, \& Devlin, 2002; Rogers \& McClelland, 2004; Vigliocco, Vinson,
Lewis, \& Garrett, 2004), aphasias (Vinson, Vigliocco, Cappa, \& Siri,
2003), probability and linguistics (Cree \& McRae, 2003; McRae, De Sa,
\& Seidenberg, 1997; Pexman, Holyk, \& Monfils, 2003), valence (Dodds,
Harris, Kloumann, Bliss, \& Danforth, 2011; Vo et al., 2009; Warriner,
Kuperman, \& Brysbaert, 2013), and reading speeds and priming (Balota et
al., 2007; Cohen-Shikora, Balota, Kapuria, \& Yap, 2013; Hutchison et
al., 2013; Keuleers, Lacey, Rastle, \& Brysbaert, 2012) to name a small
subset of research avenues. Big data has manifested in psycholinguistics
over the last decade in the form of grant funded megastudies to collect
and analyze large text corpora (the SUBTLEX projects) or to examine
numerous word properties in one study (the Lexicon projects). The
SUBTLEX projects were designed to analyze frequency counts for concepts
across extremely large corpora sizes using subtitles as a substitute for
natural speech. The investigation of these measures was first spurred by
the realization that word frequency is an important predictor of naming
and lexical decision times (Balota, Cortese, Sergent-Marshall, Spieler,
\& Yap, 2004; Rayner \& Duffy, 1986). While previous measures of
frequency (i.e. Baayen, Piepenbrock, Gulikers, \& Linguistic Data
Consortium, n.d.; Burgess \& Livesay, 1998; Kucera \& Francis, 1967)
were based on large 1 million + word corpora, they were poor predictors
of reaction times (Balota et al., 2004; Brysbaert \& New, 2009; Zevin \&
Seidenberg, 2002). Further, it appears from Brysbaert and New (2009)'s
investigation into corpora size and type, that not only should the
corpora be large (\textgreater{}16 million), but the underlying source
of the text data matters (Internet versus subtitles), as well as the
contextual diversity of the data (i.e.~number of occurrences across
sources; Adelman, Brown, \& Quesada, 2006). Not only has their work
(Brysbaert \& New, 2009) been included in newer lexical studies
(Hutchison et al., 2013; Yap, Tan, Pexman, \& Hargreaves, 2011), but
SUBTLEX projects have been published in Dutch (Keuleers, Brysbaert, \&
New, 2010), Greek {[}Dimitropoulou2010{]}, Spanish (Cuetos, Glez-Nosti,
Barbon, \& Brysbaert, 2011), Chinese (Cai \& Brysbaert, 2010), French
(New, Brysbaert, Veronis, \& Pallier, 2007), British English (Heuven,
Mandera, Keuleers, \& Brysbaert, 2014), and German (Brysbaert et al.,
2011). The Lexicon projects created large databases of validated mono-
and multisyllabic words to assist in the creation of controlled
experimental stimuli sets for future experiments. These databases
contain lexical decision and naming response times, as well as typical
word confound variables such as orthographic neighborhood, phonological
and morphological characteristics. While the English Lexicon Project
(Balota et al., 2007) is the most cited of the lexicons, other languages
include Chinese (Sze, Rickard Liow, \& Yap, 2014), Malay (Yap, Rickard
Liow, Jalil, \& Faizal, 2010), Dutch (Keuleers et al., 2010), and
British English (Keuleers et al., 2012). Another twenty or so similar
lexical database publications can be found in the literature covering
French (Lété \& Sprenger-Charolles, 2004), Italian (Barca, Burani, \&
Arduino, 2002), Arabic (Boudelaa \& Marslen-Wilson, 2010), and
Portuguese (Soares et al., 2014). The availability of big data has
augmented the psycholinguistic literature, but these projects are
certainly time consuming due to the amount of participant data required
to achieve reliable and stable norms. The solution potentially lies in
several avenues of easily obtainable data. First, Amazon's Mechanical
Turk, an online crowdsourcing avenue that allows researchers to pay
users to complete questionnaires, has shown to be a reliable, diverse
participant pool made available at very low cost (Buhrmester et al.,
2011; Mason \& Suri, 2012). Researchers can pre-screen for specific
populations, as well as post-screen surveys for incomplete or
inappropriate responses, thus saving time and money with the elimination
of poor data. Because of the popularity of Mechanical Turk, large
amounts of data can be collected in shorter time periods than
traditional experiments. Mechanical Turk has been used to collect data
for semantic word pair norms (Buchanan, Holmes, Teasley, \& Hutchison,
2013), age of acquisition ratings (Kuperman, Stadthagen-Gonzalez, \&
Brysbaert, 2012), concreteness ratings (Brysbaert, Warriner, \&
Kuperman, 2014), past tense information (Cohen-Shikora et al., 2013),
and valence and arousal ratings (Dodds et al., 2011; Jasmin \&
Casasanto, 2012; Warriner et al., 2013). Additionally, in a similar vein
to the SUBTLEX projects, linguistic data has been mined from open source
data, such as the New York Times, music lyrics, and Twitter (Dodds et
al., 2011; Kloumann, Danforth, Harris, Bliss, \& Dodds, 2012). Finally,
De Deyne, Navarro, and Storms (2013) have seen success in simply setting
up a special website (www.smallworldofwords.com) to collect word pair
association norms. The evolution of big data provides exciting
opportunities for exploration into psycholinguistics, and this article
features the trends in publications of normed datasets across the
literature allowing for a large-scale picture of the developments of
trends in psychological stimuli. Historically, these norms have been
published in journals connected to the Psychonomic Society, such as
Behavior Research Methods, Psychonomic Monograph Supplements, and
Perception and Psychophysics. The society once hosted an electronic
database that contained the links to these norms, as well as a search
tool to find information about previously published works (Vaughan,
2004). The sale of the society journals to Springer publications has
improved journal visibility and user-friendly access, but also has left
a need for an indexed list of database publications that span multiple
keywords and journal websites. Therefore, the purpose of this article is
twofold: 1) to present a searchable, cataloged database of normed
stimuli and related materials for a wide range of experimental research,
and 2) to examine trends in the publications of these articles to assess
the big data movement within psycholinguistics. \#Website Readers can
find the website by going to the host site for semantic word pair norms
Buchanan et al. (2013) at \url{http://www.wordnorms.com} or by going to
the direct link at \url{http://wordnorms.missouristate.edu}. The
wordnorms.com website will redirect you to the direct link and was meant
to be an easier web address to remember, in addition to a permanent
address in case of a change in hosting university. From this page, the
top navigation bar includes a link for Norms to direct the reader to the
LAB page. The main LAB page, as shown in Figure 1 includes the purpose
statement, and several website options. From here, users can suggest
articles that should be included in the dataset by clicking on Enter
data here, view all the data in an easy to copy format (View all data
big) or in a smaller more readable format (View all data small), search
the database with large, easy to copy table (Search with large data
output) or small table formats (Search with small data output) and view
many of the tables presented here. These tables are dynamic, and they
update with each addition to the database, which allows the user to view
current statistics even after publication. Although the website was
designed to be intuitively user friendly, a complete how-to guide is
included online for unfamiliar users. Specific features will be outlined
below in relation to the database creation.

\section{Database}\label{database}

\section{Methods}\label{methods}

\subsection{Materials}\label{materials}

Bradshaw (1984) and Proctor and Kim-Phuong (1999)'s lists of database
information were used as starting points for collection of research
articles. We searched Academic Search Premier, PsycInfo, and ERIC
through the EBSCO host system, as well as Google scholar to find other
relevant articles using the following keywords: corpus, linguistic
database, linguistic norms, norms, and database. Additionally, since
many of the original articles were hosted by the Psychonomic Society,
the Springer website was searched with these terms that covered the
newer editions of Behavior Research Methods and Memory \& Cognition. We
then filtered for articles that met the following criteria: 1) contained
database information as supplemental material, 2) demonstrated programs
related to building research stimuli using normed databases, or 3)
generated new calculations of lexical variables. Research articles that
used normed databases in experimental design or tested those variables
validity/reliability were excluded if they did not include new database
information. Additional articles were found while coding initial
publications by searching citations for stimuli selection. For example,
the Snodgrass and Vanderwart (1980) norms were cited in many newer
articles on line drawings, and therefore this article was subsequently
entered into the database. At the time of writing, 561 articles, books,
websites and technical reports were included in the following analyses.
\#\#Coding Procedure The tables with summaries from Bradshaw (1984) and
Proctor and Kim-Phuong (1999) were consulting for a starting point for
data coding. Then, the first round of articles found (approximately 100)
from the methods described above were analyzed to determine information
that would be pertinent to a user who wished to search for normed
stimuli. Based on these reviews and lab discussions, we coded the
following information from each article: 1) journal information, 2)
stimuli types, 3) stimuli language, 4) programs or corpus name, 5)
keywords, which we refer to as tags, 6) special populations, and 7)
other notes that did not fit into those categories. Each piece of
information is detailed below. In some instances, codes were not used as
frequently as expected based on these initial discussions, but were
included to allow more specificity in searching, as well as the
flexibility to include those options for articles subsequently added to
the database. \#\#\#Journal Information Each article was coded with the
citation information, and a complete list of citations can be found on
the website portal by clicking on view all data. All author last names
are listed, along with publication year, article title, journal title,
volume, page numbers, and digital object identifier (DOI) when
available. This information is listed in citation format in the small
table output, and separated into columns in the large table output for
easier sorting and searching. For newer articles that have been
published online first without volume or page numbers, X and XX-XX are
used as placeholders until official publication. DOIs in all table
outputs are hyperlinked to the article on the publication journal's
website, which is accessible if the user has access through their
membership with a professional organization or university. Although APA
style dictates et al. for references after the second author or
immediately for large author publications, all names were included each
time they were referenced (see below). The inclusion of these names
allows a user to search for specific researchers, as well as separates
different publications by the same first author. A complete list of
publication sources, number of times cited, and percentages can be found
online by clicking Journal Frequency Table. \#\#\#Stimuli Types While
this publication was originally intended for linguistic database norms,
other types of experimental stimuli were apparent after background
review. Therefore, stimuli were coded based on the dominant description
from the article (i.e.~although heteronyms are words and word pairs,
they were coded specifically as heteronyms). The number of stimuli
presented in the appendix or database was coded with the stimuli, unless
the article covered specific programs, search or experimental creation
tools (the majority of the other category). Because many articles
included two types of stimuli, or references to different articles where
stimuli were selected from, two options for stimuli were included.
Therefore, the total values for number of stimuli do not add up to the
number of articles in the database because of multiple instances in
articles. Table 1 includes a stimuli list, the number of times that each
stimuli was used, percentage of the total stimuli codes, the mean and
standard deviation of the number of those stimuli, minimum/maximum
values, and a brief variable description. Researchers often cited
specific previous works where stimuli were selected from, and these
references were included in the stimuli column. Further, if stimuli were
associated with a specific database (such as CELEX: Baayen et al., n.d.
and ANEW: Bradley and Lang (1999)), those abbreviations were included
for searching capabilities. Table 1 is included dynamically online, such
that updates are included automatically, and can be found under the
Stimuli Frequency Table link. \#\#\#Stimuli Language The language of the
stimuli set was coded by starting with the most common languages from
the first articles surveyed, and others were added as it was apparent
that several norms were present for that language (such as Japanese,
Dutch, and Greek). If the stimuli were non- linguistic selections, like
pictures and line drawings, the language of the participants used to
norm the set was used, which was commonly English. The other category
was used for low-frequency languages, as well as a multiple category for
datasets with more than one set of language norms. One potential
limitation of the LAB was that English is the first language for the
authors; however, translation tools were used to code sources found in
other languages. The LAB portal includes options to report errors in
coding, as well as a form to enter new articles that may have been
missed due to this drawback. Table 2 shows language frequencies and
percentages, and the online version can be found by clicking the
Language Frequency Table link. \#\#\#Program/corpus name In many
instances, megastudies are often named, such as the English Lexicon
Project (Balota et al., 2007), for easier reference. This information
was included in the in the dataset, which will also help researchers
with the stimuli references as described above. For example, a newer
study may reference using the BOSS database (Brodeur, Dionne-Dostie,
Montreuil, \& Lepage, 2010), and having that information would make
searching for the original article easier by using the corpus name
column (especially in instances the dataset name is not listed in the
article title). The names of programs or tools were also entered, such
as NIM (Guasch, Boada, Ferré, \& Sánchez-Casas, 2013), a new stimuli
selection tool for psycholinguistic studies. \#\#\#Tags Keyword tags are
the majority of the database, as they allow for the best understanding
of trends and availability of stimuli. Table 3 shows a list of tags,
frequencies, percentages, descriptions, and correlations (described
below). Each article was coded with tags based on the description of the
accessible data, so that one article may have many tags. However, due to
the cumulative nature of database research, this tagging system does not
mean that each article collected that particular type of data. The most
common example of this distinction occurs when data is combined across
sources, but presented in a new article. The Maki, McKinley, and
Thompson (2004) semantic distance norms also included values from the
South Florida Free Association norms (Nelson, McEvoy, \& Schreiber,
2004), and Latent Semantic Analysis (Landauer \& Dumais, 1997).
Therefore, this article was coded with association and semantics, even
though the association norms were not collected in that paper. As
described above, some small frequency tags were used because of the
initial pass through newer articles, but these were left in the database
because of their specificity, and they can be used in future additions.
\#\#\#Special Populations While coding articles, it became apparent that
a subset of the normed data was tested on specific special populations.
Consequently, demographic data such as gender, age, ethnicity, and grade
school year were listed as described in the article (i.e.~if ages were
used, age was listed, but if grade year was used, it was listed rather
than translating to specific ages). \#\#\#Other/Notes Lastly, places for
more description were included for tags or variables not frequently
used, which was especially useful for program descriptions, as well as
descriptions of specific types of stimuli (i.e.~CVC trigrams). In
several instances, notes that appeared frequently were moved to tags
(such as similarity) after the database had several hundred articles
sampled. All information described above without a specific table
(special populations, other, program/corpus names, and journal
information) can be found by clicking on either the small or large view
data links online.

\section{Results and Discussion}\label{results-and-discussion}

\subsubsection{Journals}\label{journals}

Journal results, unsurprisingly, show that the wealth of data was
published in Behavior Research Methods (59.42\% combined across name
changes). However, a large number of articles also appeared in
Psychonomic Monograph Supplements (3.41\%), Journal of Verbal Learning
and Verbal Behavior (2.69\%), Psychonomic Science (2.69\%), Journal of
Experimental Psychology (combined across subjournals, 2.34\%),
Perception \& Psychophysics (2.33\%), Memory \& Cognition (2.15\%),
Bulletin of the Psychonomic Society (1.26\%), Norms of Word Association
(1.44\%; Postman \& Keppel, 1970), and a large category of other
publications (21.24\%). The complete list can be found in the Journal
Frequency Table online, as there were 107 different entries for
journals, books, and websites of publications. While some of these
sources were not published with peer review (i.e.~websites), they were
generally found through citations of other peer-reviewed work. Although
Behavior Research Methods has dominated the field for publications, the
large array of options for publishing indicates a growth in the
available avenues for researchers in this field (for example, open
source journals such as PLoS ONE and websites). Figure 2 portrays the
number of publications in half decade intervals, and there has been a
clear expansion of database and program papers, as part of the growth in
big data. Interestingly, a first growth of publications tracks with the
1950s cognitive revolution (Miller, 2003), but an odd decline in
publications occurred from the 1970s to 1990s. The last twenty years has
shown unbelievable progress in this area, at over 121 publications in
the last four years alone (2010-2013). This chart can be found in
greater detail online, under the Papers Per Year Graph link, showing the
ups and downs of publications by year in a larger format. For example,
2004, 2010 and 2013 were big years for linguistic publications, while
2005 and 2006 had smaller numbers of publications. Even with these
fluctuations, a clear growth curve in publications can be found since
the 90s. \#\#\#Stimuli Stimuli are presented in Table 1 with totals,
percentages, means (standard deviations), and minimum/maximum values
(and online under Stimuli Frequency Table). Naturally, the publication
of word norms was over half the dataset (50.55\%), which has quite a
large range of quantity of stimuli from only ten words to a large corpus
of over 500 million words. Other types of word stimuli also appear
commonly in the stimuli set such as categories, letters, and word pairs.
Because linguistic data was of interest to this publication, we selected
publications based on words, and plotted the number of stimuli presented
in the paper to examine big data developments. These data were broken
down by set size in Figure 3. The upper left hand quadrant shows all
stimuli across years, and the big data publications stand out in the
last fifteen years of publications. This data was then further broken
down into small datasets (\textless{}1,000 stimuli; upper right
quadrant), medium datasets (1,000 -- 500,000 stimuli; bottom left
quadrant), and large datasets (500,000 + although there is a large jump
between medium and large as most data is either half million or less or
a million or more; bottom right quadrant). The small dataset graph shows
that these publications are common across time, while the bottom two
quadrants are more telling for the megastudies trend investigation. As
with languages and tags (below), we see an increase in the number of
medium and very large datasets across the years (the lone outlier in the
large dataset is the Brown Corpus, Kucera \& Francis, 1967).
\#\#\#Languages The variety and number of languages for stimuli provided
an encouraging picture of the growth and diversity of psycholinguistic
stimuli, as seen in Table 2. Many articles include multiple languages
(4.10\%), as well as the inclusion of both Portuguese (1.78\%) and
Spanish (6.60\%), French (5.70\%), and even a small number of articles
in Greek (0.89\%). To examine trends, the English only articles were
filtered out of the dataset since they were the majority of publications
(63.99\%), and were published across all years present in this data.
More than half of the papers with multiple languages have been published
since 2010 (56.52\%). Additionally, the last ten years have seen an
explosion of publications in non-English languages (134 publications,
74.81\%) with nearly double the number in 2013 (16) than 2012 (9).
\#\#\#Tags Table 3 displays the number, percentages, correlations of
tags across (with?) year, and descriptions of tags (also found online
under Tag Frequency Table with N and \% values based on any updated
numbers). Undoubtedly, these tags represent changes in terminology over
time, and some could be combined or recoined. However, even if low
frequency (N \textless{} 10; fourteen tags) tags are excluded,
thirty-five different tags were used to describe the types of
psycholinguistic data. Many of these tags can be considered individual
research areas, and the sizeable number of different options indicates
how complex and diverse the field has become since the publication of
free association norms in 1910 (Kent \& Rosanoff association norms). The
total number of tags for each publication was then tallied, and an
average number of tags per half decade were plotted in Figure 4 to
determine if the number of variables included in a study has grown over
time (total tags and year r = 0.26, p = 0.10). Considering the larger
number of publications in the 2000s versus 1950s to 1970s, it appears
that the number of keywords for articles is also slowly growing over
time. This trend may indicate the evolution in computing possibilities
to be able to publish large amounts of data, but also may indicate a
desire to combine datasets so that even more stimuli may be considered
at once for modeling or experiment creation. Next, tags with at least a
sample of 30 publications were investigated individually for trends
across time (correlations presented in Table 3, all ps = 0.29-0.40).
Individual graphs can be created by clicking on the Tags Per Year Graph
link online, which show the total frequency of the selected tag by year.
While correlations were not significant, some small positive trends were
found, such as the increase in age of acquisition, familiarity, imagery,
orthographic neighborhood, syllables, and valence norms. Intriguingly,
meaningfulness and association both showed negative correlations, but
these correlations can be understood as an artifact of the publication
of a book on association norms in the 1970s (Postman \& Keppel, 1970),
as well as a recent drop off of in the small but steady use of
meaningfulness. These small correlations may partially be explained by
the sheer number and variation of data available in the LAB portal, as
one would expect the number of frequency tags to increase with the
recent SUBTLEX publications. Indeed, if the frequency tags are plotted
by year an increase across the last decade (22 tags in 2013 alone) can
be shown. Readers are encouraged to view the individual graphs for tags
to investigate the change of keyword frequency over time, including the
rise and demise of several research areas. For example, confusion
matrices heyday appears to range from the early 70s to the mid 80s,
while arousal norms do not make an appearance until the late 90s.
\#Conclusion This article had two main purposes: 1) to present the LAB
dataset and portal as an annotated bibliography and searchable tool for
researchers, and 2) to view trends in psycholinguistic research with an
eye toward big data. We believe the LAB website will be a useful channel
for all levels of researchers, from graduate students looking for
experimental stimuli to design their experiments, to the familiar
investigator who wishes to dig deeper into the diverse choices offered.
Further, while the majority of publications occur in one particular
journal, the LAB allows someone to find articles they may have missed in
other areas with the advantage of being collected into one location.
User-friendly search tools are provided to aide in searching for
specific languages, stimuli, or keywords, as well as multiple outputs
for easy copying into Excel or SPSS. With the advent of DOI, links are
provided to the original source for a quick transition to the materials
provided in that publication. While this article's statistics will
become dated with the updates to the LAB, dynamic tables and graphs are
provided online to see the current status of the field. Lastly, we
encourage users to actively report errors and suggest updates for the
LAB dataset as a way to crowd source information that is surely missing,
especially in non-English languages. In the introduction, we provided
two examples of current megastudies (SUBTLEX and the Lexicon projects),
in addition to how researchers might collect big data through Mechanical
Turk or Twitter. This article stepped back from looking at individual,
large studies or ways to collect data to use the information provided by
publications as a window into the fluctuations of the field. Megastudies
have become a prevalent topic, but data could have revealed that this
popularity was due to recent publication of a small subset of articles.
Instead, analyses showed that not only are the numbers of publications
accumulating, but the sizes of datasets are also growing in tandem.
Megastudies specifically focus on large datasets, but big data can also
be indicated here by the divergence in languages available, number of
places to publish such data, and the increasing number of keywords for
articles across years. Time will tell if these trends can and will
continue or if certain areas will see a confusion matrix type decline
after many large datasets are published. With the move of traditional
lab experiments to smartphone and tablet technology (Dufau et al.,
2011), it seems likely that researchers in psycholinguistics will
continue to find new and creative ways to modernize the field.

\newpage

\section{References}\label{references}

\setlength{\parindent}{-0.5in} \setlength{\leftskip}{0.5in}

\hypertarget{refs}{}
\hypertarget{ref-Adelman2006}{}
Adelman, J. S., Brown, G. D. A., \& Quesada, J. F. (2006). Contextual
Diversity Not Word Frequency Determines Time To Read. \emph{Psychology},
\emph{17}(Cd), 814--823. Retrieved from
\url{http://pss.sagepub.com/content/17/9/814.short}

\hypertarget{ref-Baayen}{}
Baayen, R. H., Piepenbrock, R., Gulikers, L., \& Linguistic Data
Consortium. (n.d.). The CELEX Lexical Database (CD-ROM). Philadelphia,
PA:

\hypertarget{ref-Balota2004}{}
Balota, D. A., Cortese, M. J., Sergent-Marshall, S. D., Spieler, D. H.,
\& Yap, M. J. (2004). Visual word recognition of single-syllable words.
\emph{Journal of Experimental Psychology: General}, \emph{133}(2),
283--316.
doi:\href{https://doi.org/10.1037/0096-3445.133.2.283}{10.1037/0096-3445.133.2.283}

\hypertarget{ref-Balota2007}{}
Balota, D. A., Yap, M. J., Cortese, M. J., Hutchison, K. A., Kessler,
B., Loftis, B., \ldots{} Treiman, R. (2007). The english lexicon
project. \emph{Behavior Research Methods}, \emph{39}(3), 445--459.
doi:\href{https://doi.org/10.3758/BF03193014}{10.3758/BF03193014}

\hypertarget{ref-Barca2002}{}
Barca, L., Burani, C., \& Arduino, L. S. (2002). Word naming times and
psycholinguistic norms for Italian nouns. \emph{Behavior Research
Methods, Instruments, \& Computers : A Journal of the Psychonomic
Society, Inc}, \emph{34}(3), 424--434.
doi:\href{https://doi.org/10.3758/BF03195471}{10.3758/BF03195471}

\hypertarget{ref-Boudelaa2010}{}
Boudelaa, S., \& Marslen-Wilson, W. D. (2010). Aralex: A lexical
database for modern standard Arabic. \emph{Behavior Research Methods},
\emph{42}(2), 481--487.
doi:\href{https://doi.org/10.3758/BRM.42.2.481}{10.3758/BRM.42.2.481}

\hypertarget{ref-Bradley1999}{}
Bradley, M. M., \& Lang, P. J. (1999). \emph{Affective Norms for English
Words (ANEW): Instruction Manual and Affective Ratings} (No. C-1). The
Center for Research in Psychophysiology, University of Florida.

\hypertarget{ref-Bradshaw1984}{}
Bradshaw, J. L. (1984). A guide to norms, ratings, and lists.
\emph{Memory \& Cognition}, \emph{12}(2), 202--206.
doi:\href{https://doi.org/10.3758/BF03198435}{10.3758/BF03198435}

\hypertarget{ref-Brodeur2010}{}
Brodeur, M. B., Dionne-Dostie, E., Montreuil, T., \& Lepage, M. (2010).
The bank of standardized stimuli (BOSS), a new set of 480 normative
photos of objects to be used as visual stimuli in cognitive research.
\emph{PLoS ONE}, \emph{5}(5).
doi:\href{https://doi.org/10.1371/journal.pone.0010773}{10.1371/journal.pone.0010773}

\hypertarget{ref-Brysbaert2009}{}
Brysbaert, M., \& New, B. (2009). Moving beyond Kučera and Francis: A
critical evaluation of current word frequency norms and the introduction
of a new and improved word frequency measure for American English.
\emph{Behavior Research Methods}, \emph{41}(4), 977--990.
doi:\href{https://doi.org/10.3758/BRM.41.4.977}{10.3758/BRM.41.4.977}

\hypertarget{ref-Brysbaert2011}{}
Brysbaert, M., Buchmeier, M., Conrad, M., Jacobs, A. M., Bölte, J., \&
Böhl, A. (2011). The word frequency effect: A review of recent
developments and implications for the choice of frequency estimates in
German. \emph{Experimental Psychology}, \emph{58}(5), 412--424.
doi:\href{https://doi.org/10.1027/1618-3169/a000123}{10.1027/1618-3169/a000123}

\hypertarget{ref-Brysbaert2013}{}
Brysbaert, M., Warriner, A. B., \& Kuperman, V. (2014). Concreteness
ratings for 40 thousand generally known English word lemmas.
\emph{Behavior Research Methods}, \emph{46}(3), 904--911.
doi:\href{https://doi.org/10.3758/s13428-013-0403-5}{10.3758/s13428-013-0403-5}

\hypertarget{ref-Buchanan2013}{}
Buchanan, E. M., Holmes, J. L., Teasley, M. L., \& Hutchison, K. A.
(2013). English semantic word-pair norms and a searchable Web portal for
experimental stimulus creation. \emph{Behavior Research Methods},
\emph{45}(3), 746--757.
doi:\href{https://doi.org/10.3758/s13428-012-0284-z}{10.3758/s13428-012-0284-z}

\hypertarget{ref-Buhrmester2011}{}
Buhrmester, M., Kwang, T., Gosling, S. D., Buhrmester, M., Kwang, T., \&
Gosling, S. D. (2011). Amazon's Mechanical Turk: A New Source of
Inexpensive, Yet High-Quality, Data?, \emph{6}(1), 3--5.

\hypertarget{ref-Burgess1998}{}
Burgess, C., \& Livesay, K. (1998). The effect of corpus size in
predicting reaction time in a basic word recognition task: Moving on
from Kučera and Francis. \emph{Behavior Research Methods, Instruments,
and Computers}, \emph{30}(2), 272--277.
doi:\href{https://doi.org/10.3758/BF03200655}{10.3758/BF03200655}

\hypertarget{ref-Cai2010}{}
Cai, Q., \& Brysbaert, M. (2010). SUBTLEX-CH: Chinese word and character
frequencies based on film subtitles. \emph{PLoS ONE}, \emph{5}(6).
doi:\href{https://doi.org/10.1371/journal.pone.0010729}{10.1371/journal.pone.0010729}

\hypertarget{ref-Cohen-Shikora2013}{}
Cohen-Shikora, E. R., Balota, D. A., Kapuria, A., \& Yap, M. J. (2013).
The past tense inflection project (PTIP): Speeded past tense
inflections, imageability ratings, and past tense consistency measures
for 2,200 verbs. \emph{Behavior Research Methods}, \emph{45}(1),
151--159.
doi:\href{https://doi.org/10.3758/s13428-012-0240-y}{10.3758/s13428-012-0240-y}

\hypertarget{ref-Cree2003}{}
Cree, G. S., \& McRae, K. (2003). Analyzing the Factors Underlying the
Structure and Computation of the Meaning of Chipmunk, Cherry, Chisel,
Cheese, and Cello (and many Other Such Concrete Nouns). \emph{Journal of
Experimental Psychology: General}, \emph{132}(2), 163--201.
doi:\href{https://doi.org/10.1037/0096-3445.132.2.163}{10.1037/0096-3445.132.2.163}

\hypertarget{ref-Cree1999}{}
Cree, G. S., McRae, K., \& McNorgan, C. (1999). An attractor model of
lexical conceptual processing: Simulating semantic priming.
\emph{Cognitive Science}, \emph{23}, 371--414.
doi:\href{https://doi.org/10.1016/S0364-0213(99)00005-1}{10.1016/S0364-0213(99)00005-1}

\hypertarget{ref-Cuetos2011}{}
Cuetos, F., Glez-Nosti, M., Barbon, A., \& Brysbaert, M. (2011).
SUBTLEX-ESP: Spanish word frequencies based on film subtitles.
\emph{Psicologica}, \emph{32}, 133--143.

\hypertarget{ref-DeDeyne2013}{}
De Deyne, S., Navarro, D. J., \& Storms, G. (2013). Better explanations
of lexical and semantic cognition using networks derived from continued
rather than single-word associations. \emph{Behavior Research Methods},
\emph{45}(2), 480--498.
doi:\href{https://doi.org/10.3758/s13428-012-0260-7}{10.3758/s13428-012-0260-7}

\hypertarget{ref-Dodds2011}{}
Dodds, P. S., Harris, K. D., Kloumann, I. M., Bliss, C. A., \& Danforth,
C. M. (2011). Temporal patterns of happiness and information in a global
social network: Hedonometrics and Twitter. \emph{PLoS ONE},
\emph{6}(12).
doi:\href{https://doi.org/10.1371/journal.pone.0026752}{10.1371/journal.pone.0026752}

\hypertarget{ref-Dufau2011}{}
Dufau, S., Duñabeitia, J. A., Moret-Tatay, C., McGonigal, A., Peeters,
D., Alario, F. X., \ldots{} Grainger, J. (2011). Smart phone, smart
science: How the use of smartphones can revolutionize research in
cognitive science. \emph{PLoS ONE}, \emph{6}(9), 9--11.
doi:\href{https://doi.org/10.1371/journal.pone.0024974}{10.1371/journal.pone.0024974}

\hypertarget{ref-Guasch2013}{}
Guasch, M., Boada, R., Ferré, P., \& Sánchez-Casas, R. (2013). NIM: A
Web-based Swiss army knife to select stimuli for psycholinguistic
studies. \emph{Behavior Research Methods}, \emph{45}(3), 765--771.
doi:\href{https://doi.org/10.3758/s13428-012-0296-8}{10.3758/s13428-012-0296-8}

\hypertarget{ref-VanHeuven2014}{}
Heuven, W. J. van, Mandera, P., Keuleers, E., \& Brysbaert, M. (2014).
SUBTLEX-UK: A new and improved word frequency database for British
English. \emph{Quarterly Journal of Experimental Psychology},
\emph{67}(6), 1176--1190.
doi:\href{https://doi.org/10.1080/17470218.2013.850521}{10.1080/17470218.2013.850521}

\hypertarget{ref-Hutchison2013}{}
Hutchison, K. A., Balota, D. A., Neely, J. H., Cortese, M. J.,
Cohen-Shikora, E. R., Tse, C.-S., \ldots{} Buchanan, E. M. (2013). The
semantic priming project. \emph{Behavior Research Methods},
\emph{45}(4), 1099--1114.
doi:\href{https://doi.org/10.3758/s13428-012-0304-z}{10.3758/s13428-012-0304-z}

\hypertarget{ref-Jasmin2012}{}
Jasmin, K., \& Casasanto, D. (2012). The QWERTY Effect: How typing
shapes the meanings of words. \emph{Psychonomic Bulletin \& Review},
\emph{19}(3), 499--504.
doi:\href{https://doi.org/10.3758/s13423-012-0229-7}{10.3758/s13423-012-0229-7}

\hypertarget{ref-Keuleers2010}{}
Keuleers, E., Brysbaert, M., \& New, B. (2010). SUBTLEX-NL: A new
measure for Dutch word frequency based on film subtitles. \emph{Behavior
Research Methods}, \emph{42}(3), 643--650.
doi:\href{https://doi.org/10.3758/BRM.42.3.643}{10.3758/BRM.42.3.643}

\hypertarget{ref-Keuleers2012}{}
Keuleers, E., Lacey, P., Rastle, K., \& Brysbaert, M. (2012). The
British Lexicon Project: Lexical decision data for 28,730 monosyllabic
and disyllabic English words. \emph{Behavior Research Methods},
\emph{44}(1), 287--304.
doi:\href{https://doi.org/10.3758/s13428-011-0118-4}{10.3758/s13428-011-0118-4}

\hypertarget{ref-Kloumann2012}{}
Kloumann, I. M., Danforth, C. M., Harris, K. D., Bliss, C. A., \& Dodds,
P. S. (2012). Positivity of the English language. \emph{PLoS ONE},
\emph{7}(1), 0--6.
doi:\href{https://doi.org/10.1371/journal.pone.0029484}{10.1371/journal.pone.0029484}

\hypertarget{ref-Kucera1967}{}
Kucera, H., \& Francis, W. N. (1967). \emph{Computational analysis of
present-day American English.} Providence, RI: Brown University Press.

\hypertarget{ref-Kuperman2012}{}
Kuperman, V., Stadthagen-Gonzalez, H., \& Brysbaert, M. (2012).
Age-of-acquisition ratings for 30,000 English words. \emph{Behavior
Research Methods}, \emph{44}(4), 978--990.
doi:\href{https://doi.org/10.3758/s13428-012-0210-4}{10.3758/s13428-012-0210-4}

\hypertarget{ref-Landauer1997}{}
Landauer, T. K., \& Dumais, S. T. (1997). A solution to Plato's problem:
The latent semantic analysis theory of acquisition, induction, and
representation of knowledge. \emph{Psychological Review}, \emph{104}(2),
211--240.
doi:\href{https://doi.org/10.1037//0033-295X.104.2.211}{10.1037//0033-295X.104.2.211}

\hypertarget{ref-Lete2004}{}
Lété, B., \& Sprenger-Charolles, L. (2004). MANULEX: A lexical database
from French readers. \emph{Behaviour Research Methods Instruments and
Computers}, \emph{36}(1), 156--166.

\hypertarget{ref-Maki2004}{}
Maki, W. S., McKinley, L. N., \& Thompson, A. G. (2004). Semantic
distance norms computed from an electronic dictionary (WordNet).
\emph{Behavior Research Methods, Instruments, \& Computers},
\emph{36}(3), 421--431.
doi:\href{https://doi.org/10.3758/BF03195590}{10.3758/BF03195590}

\hypertarget{ref-Mason2012}{}
Mason, W., \& Suri, S. (2012). Conducting behavioral research on
Amazon's Mechanical Turk. \emph{Behavior Research Methods},
\emph{44}(1), 1--23.
doi:\href{https://doi.org/10.3758/s13428-011-0124-6}{10.3758/s13428-011-0124-6}

\hypertarget{ref-McRae1997}{}
McRae, K., De Sa, V. R., \& Seidenberg, M. S. (1997). On the Nature and
Scope of Featural Representations of Word Meaning. \emph{Journal of
Experimental Psychology: General}, \emph{126}(2), 99--130.
doi:\href{https://doi.org/10.1037/0096-3445.126.2.99}{10.1037/0096-3445.126.2.99}

\hypertarget{ref-Miller2003}{}
Miller, G. A. (2003). The cognitive revolution: A historical
perspective. \emph{Trends in Cognitive Sciences}, \emph{7}, 141--144.
doi:\href{https://doi.org/10.1016/S1364-6613(03)00029-9}{10.1016/S1364-6613(03)00029-9}

\hypertarget{ref-Moss2002}{}
Moss, H. E., Tyler, L. K., \& Devlin, J. T. (2002). The emergence of
category-specific deficits in a distribuited semantic system.

\hypertarget{ref-Nelson2004}{}
Nelson, D. L., McEvoy, C. L., \& Schreiber, T. A. (2004). The University
of South Florida free association, rhyme, and word fragment norms.
\emph{Behavior Research Methods, Instruments, \& Computers},
\emph{36}(3), 402--407.
doi:\href{https://doi.org/10.3758/BF03195588}{10.3758/BF03195588}

\hypertarget{ref-New2007}{}
New, B., Brysbaert, M., Veronis, J., \& Pallier, C. (2007). The use of
film subtitles to estimate word frequencies. \emph{Applied
Psycholinguistics}, \emph{28}(4), 661--677.
doi:\href{https://doi.org/10.1017/S014271640707035X}{10.1017/S014271640707035X}

\hypertarget{ref-Pexman2003}{}
Pexman, P. M., Holyk, G. G., \& Monfils, M.-H. (2003).
Number-of-features effects and semantic processing. \emph{Memory \&
Cognition}, \emph{31}(6), 842--855.
doi:\href{https://doi.org/10.3758/BF03196439}{10.3758/BF03196439}

\hypertarget{ref-Postman1970}{}
Postman, L., \& Keppel, G. (1970). \emph{Norms of word association.} New
York: Academic Press.

\hypertarget{ref-Proctor1999}{}
Proctor, R. W., \& Kim-Phuong, L. V. (1999). Index of norms and ratings
published in the Psychonomic Society journals. \emph{Behavior Research
Methods, Instruments, and Computers}, \emph{31}(4), 659--667.
doi:\href{https://doi.org/10.3758/BF03200742}{10.3758/BF03200742}

\hypertarget{ref-Rayner1986}{}
Rayner, K., \& Duffy, S. A. (1986). Lexical complexity and fixation
times in reading: Effects of word frequency, verb complexity, and
lexical ambiguity. \emph{Memory \& Cognition}, \emph{14}(3), 191--201.
doi:\href{https://doi.org/10.3758/BF03197692}{10.3758/BF03197692}

\hypertarget{ref-Rogers2004}{}
Rogers, T. T., \& McClelland, J. L. (2004). \emph{Semantic cognition: A
parallel distributed processing approach}. MIT Press.

\hypertarget{ref-Snodgrass1980}{}
Snodgrass, J. G., \& Vanderwart, M. (1980). A standardized set of 260
pictures: Norms for name agreement, image agreement, familiarity, and
visual complexity. \emph{Journal of Experimental Psychology: Human
Learning and Memory}, \emph{6}(2), 174--215.
doi:\href{https://doi.org/10.1037/0278-7393.6.2.174}{10.1037/0278-7393.6.2.174}

\hypertarget{ref-Soares2014}{}
Soares, A. P., Medeiros, J. C., Simões, A., Machado, J., Costa, A.,
Iriarte, Á., \ldots{} Comesaña, M. (2014). ESCOLEX: A grade-level
lexical database from European Portuguese elementary to middle school
textbooks. \emph{Behavior Research Methods}, \emph{46}(1), 240--253.
doi:\href{https://doi.org/10.3758/s13428-013-0350-1}{10.3758/s13428-013-0350-1}

\hypertarget{ref-Sze2014}{}
Sze, W. P., Rickard Liow, S. J., \& Yap, M. J. (2014). The Chinese
Lexicon Project: A repository of lexical decision behavioral responses
for 2,500 Chinese characters. \emph{Behavior Research Methods},
\emph{46}(1), 263--273.
doi:\href{https://doi.org/10.3758/s13428-013-0355-9}{10.3758/s13428-013-0355-9}

\hypertarget{ref-Vaughan2004}{}
Vaughan, J. (2004). Editorial: a web-based archive of norms, stimuli,
and data. \emph{Behavior Research Methods, Instruments, \& Computers : A
Journal of the Psychonomic Society, Inc}, \emph{36}(3), 363--370.
doi:\href{https://doi.org/10.3758/BF03195583}{10.3758/BF03195583}

\hypertarget{ref-Vigliocco2004}{}
Vigliocco, G., Vinson, D. P., Lewis, W., \& Garrett, M. F. (2004).
Representing the meanings of object and action words: The featural and
unitary semantic space hypothesis. \emph{Cognitive Psychology},
\emph{48}(4), 422--488.
doi:\href{https://doi.org/10.1016/j.cogpsych.2003.09.001}{10.1016/j.cogpsych.2003.09.001}

\hypertarget{ref-Vinson2003}{}
Vinson, D. P., Vigliocco, G., Cappa, S., \& Siri, S. (2003). The
breakdown of semantic knowledge: Insights from a statistical model of
meaning representation. \emph{Brain and Language}, \emph{86}(3),
347--365.
doi:\href{https://doi.org/10.1016/S0093-934X(03)00144-5}{10.1016/S0093-934X(03)00144-5}

\hypertarget{ref-Vo2009}{}
Vo, M. L. H., Conrad, M., Kuchinke, L., Urton, K., Hofmann, M. J., \&
Jacobs, A. M. (2009). The Berlin Affective Word List Reloaded (BAWL-R).
\emph{Behavior Research Methods}, \emph{41}(2), 534--538.
doi:\href{https://doi.org/10.3758/BRM.41.2.534}{10.3758/BRM.41.2.534}

\hypertarget{ref-Warriner2013}{}
Warriner, A. B., Kuperman, V., \& Brysbaert, M. (2013). Norms of
valence, arousal, and dominance for 13,915 English lemmas.
\emph{Behavior Research Methods}, \emph{45}(4), 1191--1207.
doi:\href{https://doi.org/10.3758/s13428-012-0314-x}{10.3758/s13428-012-0314-x}

\hypertarget{ref-Yap2010}{}
Yap, M. J., Rickard Liow, S. J., Jalil, S. B., \& Faizal, S. S. B.
(2010). The malay lexicon project: A database of lexical statistics for
9,592 words. \emph{Behavior Research Methods}, \emph{42}(4), 992--1003.
doi:\href{https://doi.org/10.3758/BRM.42.4.992}{10.3758/BRM.42.4.992}

\hypertarget{ref-Yap2011}{}
Yap, M. J., Tan, S. E., Pexman, P. M., \& Hargreaves, I. S. (2011). Is
more always better? Effects of semantic richness on lexical decision,
speeded pronunciation, and semantic classification. \emph{Psychonomic
Bulletin and Review}, \emph{18}(4), 742--750.
doi:\href{https://doi.org/10.3758/s13423-011-0092-y}{10.3758/s13423-011-0092-y}

\hypertarget{ref-Zevin2002}{}
Zevin, J., \& Seidenberg, M. (2002). Age of acquisition effects in word
reading and other tasks. \emph{Journal of Memory and Language},
\emph{47}(1), 1--29.
doi:\href{https://doi.org/10.1006/jmla.2001.2834}{10.1006/jmla.2001.2834}






\end{document}
